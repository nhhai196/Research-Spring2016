%%%%%%%%%%%%%%%%%%%%%%%%%%%%%%%%%%%%%%%%%
% University Assignment Title Page 
% LaTeX Template
% Version 1.0 (27/12/12)
%
% This template has been downloaded from:
% http://www.LaTeXTemplates.com
%
% Original author:
% WikiBooks (http://en.wikibooks.org/wiki/LaTeX/Title_Creation)
%
% License:
% CC BY-NC-SA 3.0 (http://creativecommons.org/licenses/by-nc-sa/3.0/)
% 
% Instructions for using this template:
% This title page is capable of being compiled as is. This is not useful for 
% including it in another document. To do this, you have two options: 
%
% 1) Copy/paste everything between \begin{document} and \end{document} 
% starting at \begin{titlepage} and paste this into another LaTeX file where you 
% want your title page.
% OR
% 2) Remove everything outside the \begin{titlepage} and \end{titlepage} and 
% move this file to the same directory as the LaTeX file you wish to add it to. 
% Then add \input{./title_page_1.tex} to your LaTeX file where you want your
% title page.
%
%%%%%%%%%%%%%%%%%%%%%%%%%%%%%%%%%%%%%%%%%

%----------------------------------------------------------------------------------------
%	PACKAGES AND OTHER DOCUMENT CONFIGURATIONS
%----------------------------------------------------------------------------------------

\documentclass[12pt]{book}
\usepackage{amsmath,amsfonts,amssymb,amsthm,titling,url,array}
\usepackage{mathtools}

\newcommand{\defeq}{\vcentcolon=}
\newcommand{\eqdef}{=\vcentcolon}

\usepackage[margin=1.0in]{geometry}

\theoremstyle{plain}
\newtheorem{theorem}{Theorem}[chapter] % reset theorem numbering for each chapter

\theoremstyle{definition}
\newtheorem{definition}[theorem]{Definition} % definition numbers are dependent on theorem numbers
\newtheorem{lemma}[theorem]{Lemma}
\newtheorem{claim}[theorem]{Claim}

\begin{document}

\begin{titlepage}

\newcommand{\HRule}{\rule{\linewidth}{0.5mm}} % Defines a new command for the horizontal lines, change thickness here

\center % Center everything on the page
 
%----------------------------------------------------------------------------------------
%	HEADING SECTIONS
%----------------------------------------------------------------------------------------

\textsc{\LARGE Purdue University}\\[1.5cm] % Name of your university/college
\textsc{\Large CS 699}\\[0.5cm] % Major heading such as course name
\textsc{\large Spring 2016}\\[0.5cm] % Minor heading such as course title

%----------------------------------------------------------------------------------------
%	TITLE SECTION
%----------------------------------------------------------------------------------------

\HRule \\[0.4cm]
{ \huge \bfseries Research Thesis}\\[0.4cm] % Title of your document
\HRule \\[1.5cm]
 
%----------------------------------------------------------------------------------------
%	AUTHOR SECTION
%----------------------------------------------------------------------------------------

\begin{minipage}{0.4\textwidth}
\begin{flushleft} \large
\emph{Author:}\\
Hai \textsc{Nguyen} % Your name
\end{flushleft}
\end{minipage}
~
\begin{minipage}{0.4\textwidth}
\begin{flushright} \large
\emph{Instructor:} \\
Prof. Hemanta \textsc{Maji} % Supervisor's Name
\end{flushright}
\end{minipage}\\[4cm]

% If you don't want a supervisor, uncomment the two lines below and remove the section above
%\Large \emph{Author:}\\
%John \textsc{Smith}\\[3cm] % Your name

%----------------------------------------------------------------------------------------
%	DATE SECTION
%----------------------------------------------------------------------------------------

{\large \today}\\[3cm] % Date, change the \today to a set date if you want to be precise

%----------------------------------------------------------------------------------------
%	LOGO SECTION
%----------------------------------------------------------------------------------------

%\includegraphics{Logo}\\[1cm] % Include a department/university logo - this will require the graphicx package
 
%----------------------------------------------------------------------------------------

\vfill % Fill the rest of the page with whitespace

\end{titlepage}

%----------------------------------------------------------------------------------------
\chapter{Fourier Basics}
\section{Vector Space of Functions on Boolean Hyper-cube}
\begin{definition}[Inner Product]
Consider the $2^n$-dimensional vector space of all functions 
$f : \{ 0,1 \}^n \rightarrow\mathbb{R}$. 
We define an inner product on this space by 
$$\langle f, g \rangle 
\defeq \mathbb{E}[f \cdot g] 
= \frac{1}{2^n}\sum\limits_{x  \in \{ 0,1 \}^n} f(x)g(x)$$. 
\end{definition}
\section{Characteristic Functions}
\begin{definition} [Characteristic function]
For each $S \subseteq [n] = \{1,2,..., n\}$, we define the characteristic function  of $S$ as 
$${\chi}_S(x) = (-1)^{S \cdot x}, \text{where } S \cdot x = \sum\limits_{i=1}^{n} S_i \cdot x_i = \sum \limits_{i \in S} x_i $$.
\end{definition}

\begin{lemma}
For every $S \subseteq [n],$ \center
$\sum\limits_{x \in \{ 0, 1 \}^n} \chi_S(x) =
\begin{cases}
2^n      & \text{if} \ S = \emptyset \\
0        & \text{if} \ S \neq \emptyset
\end{cases}$
\end{lemma}

\begin{proof}
If $ S = \emptyset $, then $S \cdot x = 0$. So $\sum\limits_{x \in \{ 0, 1 \}^n} \chi_S(x) = \sum \limits_{x \in \{ 0, 1 \}^n} 1 = 2^n $.\\
If $S \neq \emptyset$, then there exists $k$ such that $S_k \neq 0$. Hence, \\
\begin{align*}
\sum\limits_{x \in \{ 0, 1 \}^n} \chi_S(x) 
& = \sum\limits_{x \in \{ 0, 1 \}^n} (-1)^{\sum \limits_{i \in S} x_i} \\
& = \sum\limits_{x \in \{ 0, 1 \}^n} [(-1)^{x_k} \cdot (-1)^{\sum\limits_{i \in S \setminus \{k \}} x_i}] \\
& = \sum\limits_{x_k \in \{0, 1 \} }(-1)^{x_k} 
\cdot \sum\limits_{x \setminus x_k \in \{0,1\}^{n-1}} (-1)^{\sum\limits_{i \in S \setminus \{k \}} x_i}  \\
& = \left[(-1)^0 + (-1)^1 \right] \sum\limits_{x \setminus x_k \in \{0,1\}^{n-1}} (-1)^{\sum\limits_{i \in S \setminus \{k \}} x_i} \\
& = 0
\end{align*}
\end{proof}

\begin{theorem}
For every $S, T \subseteq [n]$,
\center
$\langle \chi_S, \chi_T \rangle =
\begin{cases}
1      & \text{if} \ S = T \\
0      & \text{if} \ S \neq T
\end{cases}$
\end{theorem}

\begin{proof}
\begin{align*}
\langle \chi_S, \chi_T \rangle 
= \frac{1}{2^n}  \sum\limits_{x \in \{0,1\}^n} (-1)^{S \cdot x + T \cdot x}
= \frac{1}{2^n}  \sum\limits_{x \in \{0,1\}^n} (-1)^{(S \Delta T) \cdot x}
\end{align*}
where $\Delta$ is the symmetric different between two sets $S$ and $T$. \\
 $S \Delta T = \emptyset$ if and only if  $S = T$. Hence, our goal follows immediately from Lemma 1.3.
\end{proof}

\section{Fourier Basis}
\begin{theorem}
The set of all $\chi_S$ defines an orthonormal basis for the space of all real-valued function on $\{ 0, 1\}^n$
\end{theorem}

\begin{proof}
From Theorem 1.4, the set of all $\chi_S$ is an orthonormal set. Also, there are $2^n$ different $\chi_S$. Hence, the set of all $\chi_s$ must be an orthonormal basis for the space of all real-valued functions on $\{ 0,1 \}^n$.
%it suffices to show that the set of all $\chi_S$ is a basis. Suppose 
%$\sum\limits_{S \subseteq [n]}a_S \chi_S = \textbf{0}$. We will show that $X_S$ are %linearly independent by proving $a_S = 0$ for every $S$. Let $S$ be a subset of $[n]$, then 
%\begin{align*}
%          &= \chi_S \sum\limits_{T \subseteq [n]} a_T \chi_T 
%	       = \sum\limits_{T \subseteq [n]} a_T \langle \chi_S, \chi_T \rangle \\
%	       &= a_S \langle \chi_S, \chi_S \rangle +
%	          \sum\limits_{T \subseteq [n], T \neq S} a_T \langle \chi_S, \chi_T \rangle 
%	       = a_S  	       	  
%\end{align*}
%Also, there are $2^n$ different $X_S$. Hence, the set of all $\chi_S$ is a basis for the space of all real-valued function on $\{ 0,1 \}^n$
\end{proof}
The set of all $\chi_S$ is called the \textit{the Fourier basis}.

\section{Fourier Transform}
\begin{definition} [Fourier transform function]
For each $S \subseteq [n]$, we define the Fourier transform of $f$ as following: 
 $$\widehat{f}(S) \defeq \mathbb{E}[f \cdot \chi_S] = \langle f, \chi_S \rangle $$
\end{definition}

\begin{theorem}
The mapping $\mathcal{F} : f \rightarrow \widehat{f}$ is linear.
\end{theorem}

\begin{proof}
This follows from the properties of inner product. \center
$\langle af + bg, \chi_S \rangle = 
 \langle af, \chi_S \rangle + \langle bg, \chi_S \rangle = 
 a\langle f, \chi_S \rangle + b \langle g, \chi_s \rangle =
 a \widehat{f} + b \widehat{g} $
\end{proof}

The linear map $\mathcal{F} : f \rightarrow \widehat{f}$ is called the \textit{Fourier transform }.

\begin{theorem}
The linear map $\mathcal{F}$ is a bijection.
\end{theorem} 

\begin{proof}
Since the set of $\chi_S$ forms an orthonormal basis, 
\begin{equation}
f = \sum \limits_{S} \widehat{f}(S) \chi_S.
\end{equation}
Suppose $\mathcal{F}(f_1) = \mathcal{F}(f_2)$, i.e, $\widehat{f_1}(S) = \widehat{f_2}(S)$ for every $S$, then it is followed from equation (1.1) that $f_1 = f_2$. So $\mathcal{F}$ is injective. \\
Also, equation (1.1) implies that for every $\widehat{f}$, there exists a function $f = \sum \limits_{S} \widehat{f}(S) \chi_S $ such that $\mathcal{F}(f) = \widehat{f}$, which means that $\mathcal{F}$ is surjective. \\
Thus, $\mathcal{F}$ is a bijection as derived 
\end{proof}

\section{Convolution}
\begin{definition}
Given any two function $f$ and $g$ : $\{ 0,1 \}^n \rightarrow \mathbb{R}$, the convolution of $f*g$ :  $\{ 0,1 \}^n \rightarrow \mathbb{R}$ is defined as
$$(f*g)(x) \defeq \frac{1}{2^n} \sum\limits_{y \in \{0,1\}^n} f(x \oplus y) g(y)$$
\end{definition}

\begin{theorem}
If $X$ and $Y$ are $n$-bits random independent variables with probability distributions $f$ and $g$, respectively, then $2^n(f*g)$ is the distribution of the random variable $Z = X \oplus Y$.
\end{theorem}

\begin{proof}

\begin{align*}
Pr[Z = z] 
& = Pr[X = z \oplus Y] \\
& = \sum\limits_{y \in \{0,1\}^n} Pr[X = z \oplus y | Y = y] 	 \\
& = \sum\limits_{y \in \{0,1\}^n} Pr[X = z \oplus y] \cdot Pr[Y =y] \\
& = \sum\limits_{y \in \{0,1\}^n} f((z \oplus y) \cdot g(y) \\
& = 2^n (f *g)(z)
\end{align*}
\end{proof}
\begin{theorem}
For every $S \subseteq [n]$, 
$$\widehat{f*g}(S) = \widehat{f}(S) \cdot \widehat{g}(S)$$
\end{theorem}

\begin{proof}
\begin{align*}
\widehat{f*g}(S) 
& = \frac{1}{2^n}\sum \limits_{x} (f*g)(x) \chi_S(x) \\
& = \frac{1}{2^n} \sum \limits_{x} \left( \frac{1}{2^n} 
	\sum\limits_{y} f(x \oplus y) g(y) \right) \chi_S(x) \\
& = \frac{1}{2^{2n}} \sum\limits_x  \sum\limits_y f(x \oplus y) g(y) 
	\chi_S(x \oplus y) \chi_S(y) \\
& = \frac{1}{2^n}\sum \limits_{x} f(x \oplus y) \chi_S(x \oplus y) 
	\left( \frac{1}{2^n}\sum \limits_{y} g(y) \chi_S(y) \right)\\
& = \widehat{f}(S) \cdot \widehat{g}(S)		
\end{align*}
\end{proof}

Intuitively, the convolution $f * g$ is the product of the Fourier transforms of $f$ and $g$.
\begin{theorem}
Let $V$ be a subspace of dimension $k$ of $\{ 0, 1 \}^n$ and let $V^\perp$ be the dual of $V$. Define \\
 $$f(x) =
\begin{cases}
	\frac{1}{2^k} & \text{if} \ x \in V \\
	0             & \text{otherwise}.
\end{cases}$$. \\
Then
$$\widehat{f}(S) =
\begin{cases}
	\frac{1}{N} & \text{if} \ S \in V^\perp \\
	0             & \text{otherwise}.
\end{cases}$$.
\end{theorem}

\begin{proof}
Suppose $v_1, v_2,..., v_k $ is a basis of $V$.

\begin{claim}
$V = \langle v_1 \rangle \oplus \langle v_2 \rangle 
	 \oplus ... \oplus \langle v_k \rangle $
\end{claim}

\begin{claim}
${\langle v_1 \rangle}^\perp \cap ... \cap {\langle v_k \rangle}^\perp
= {\langle v_1, ..., v_k \rangle}^\perp = V^\perp$
\end{claim}

\noindent Let $f_i = \begin{cases}
	\frac{1}{2} & \text{if} \ S \in \langle v_i \rangle \\
	0             & \text{otherwise}
\end{cases}$, 
then $\widehat{f_i} = \begin{cases}
	\frac{1}{N} & \text{if} \ S \in {\langle v_i \rangle}^ \perp \\
	0             & \text{otherwise}.
\end{cases}$. 

\noindent From above claims, we immediately obtain following result.
\begin{claim}
$f = f_1 \oplus f_2 \oplus ... \oplus f_k$
\end{claim}

\noindent Hence,
$$\widehat{f}(S) = N^{k-1} \widehat{f_1}(S) ... \widehat{f_k}(S)$$ \\
If $S \in V^\perp$, then $S \in {\langle v_i \rangle}^\perp$ for every $i$, so $\widehat{f}(S) = N^{k-1} \cdot (\frac{1}{N})^k = \frac{1}{N}$. \\
If $S \not\in V^\perp$, then there exits some $i$ such that $S \not\in {\langle v_i \rangle}^\perp$, which implies $\widehat{f_i}(S) = 0$. Hence, $\widehat{f}(S) = 0$
\end{proof}

\section{Parseval's Identity}
Because the $\chi_S$ form an orthonormal basis, we have the following equality:
\begin{equation}
\langle f, g \rangle = \sum\limits_{S} \widehat{f}(S) \widehat{g}(S)
\end{equation}
In particular, when f = g we get Parseval's identity:
\begin{equation}
\| f \|_2^2 = \sum\limits_{S} \widehat{f}(S)^2
\end{equation}
This also implies:
\begin{equation}
\| f - g \|_2^2 = \sum\limits_{S} (\widehat{f}(S) - \widehat{g}(S))^2
\end{equation}

\chapter{Min Entropy}
Let $X = (x_0, x_1, ..., x_{N-1})$ be a distribution function of a random variable over $\{ 0, 1 \}^n$, where $N \defeq 2^n$. 
\begin{definition}[Min Entropy]
We define the min entropy of $X$ as follow.\center
$H_\infty (X) \defeq \max\limits_{i} (- \log x_i)$
\end{definition}

This implies that if $H_\infty(X) \geq k \ \text{then } \ x_i \leq \frac{1}{2^k} $ for every $0 \leq i \leq N-1$

\begin{theorem}[Collision Probability]
If we sample $X$ twice, then the probability we get the same result, denoted $Col(X)$, is 
$\sum\limits_{i = 0}^{N-1} {x_i}^2 = N \cdot \| X \|_2^2$.
\end{theorem}

\begin{definition}[Flat Distribution]
A probability distribution function $f \colon \{ 0, 1 \}^n \rightarrow (0,1)$ is a \textit{T-flat} if there $\exists \ S \subseteq \{ 0,1 \}^n$ such that $|S| = T$ and 
$f(x) = 
\begin{cases}
	\frac{1}{T} & \text{if} \ x \in S \\
	0             & \text{otherwise}
\end{cases}$
\end{definition}

\begin{lemma} For every $\alpha \geq \beta$, every $\alpha$-flat distribution can be written as the sum of $\beta$-flat distributions. 
\end{lemma}

\begin{theorem}
For every integer $k \geq 0$, if $H_\infty(X) \geq k$, then 
$X = \sum \alpha_i X_i$, where each $X_i$ is a \textit{$2^k$-flat}, $\alpha_i \in [0,1] \text{ for every} \ i$, and $\sum\limits_i \alpha_i = 1$.
\end{theorem}

\begin{proof}
Let $S$ be the set of all the probability distributions $X$ with $H_{\infty}(X) \geq k$, then S is a compact convex polytope in $\mathbb{R}^N$. The set of all $2^k$-flat distributions is the set of vertices (extreme points) of $S$. Since $S$ is a compact convex set, $S$ equals to the set of all convex combinations of its vertices. Hence, every distribution $X$ with  $H_{\infty}(X) \geq k$ can be written as a convex combination of $k$-flat distributions. 

\end{proof}

\begin{theorem}
If $H_\infty(X) \geq k$, then $Col(X) \leq \frac{1}{2^k}$.
\end{theorem}

\begin{proof}
By theorem 2.5, we can write $X$ as $X = \sum\limits_i \alpha_i X_i$, where each $X_i$ is a $2^k\text{-flat}$, $\sum \alpha_i = 1$, and $\alpha_i \in [0,1] \text{for every} \ i$. It is obvious that $Col(X_i) = \| X_i \|_2^2 = \frac{1}{2^k}$.\\
Collision functions are convex, so by Jensen's inequality, 
$$Col(X) = Col \left( \sum\limits_i \alpha_i X_i \right)
\leq \sum\limits_i \alpha_i \cdot Col(X_i)
= \sum\limits_i \alpha_i \frac{1}{2^k} 
=\frac{1}{2^k} \sum\limits_i \alpha_i 
= \frac{1}{2^k}$$
%By Cauchy-Sachwarz inequality, \\
%$\| X \|_2^2 
%= \| \sum \alpha_i X_i \|_2^2 
%\leq | \sum \alpha_i | \cdot | \sum \alpha_i {X_i}^2|
%= \sum \alpha_i \| {X_i} \|_2^2 
%= \sum \alpha_i \frac{1}{2^k}
%= \frac{1}{2^k}$ 
\end{proof}

\begin{theorem}
If $H_\infty(X) \geq k$, then $\sum\limits_{S} \widehat{X}(S)^2 \leq \frac{1}{N \cdot 2^k}$.
\end{theorem}

This follow immediately from the Parseval's identity.

\begin{definition}[Small Bias Distribution]
Let $\mathcal{D}$ be a probability distribution function over $\{ 0,1 \}^n$.
We say that $\mathcal{D}$ is $\alpha\text{-bias}$ if $\widehat{D}(S) \leq \frac{\alpha}{N}$. \\
%Fools all linear test means that if for any test $s$, a sample of $\mathcal{D}$, it returns 0 with probability $\frac{1}{2}+ \alpha$ and returns 1 with probability $\frac{1}{2} - \alpha$, where $\alpha \leq \alpha^*$
\end{definition}

%\begin{theorem}
%If $\mathcal{D}$ is a $\alpha^*\text{-bias}$, then $\widehat{\mathcal{D}}(S) \leq \frac{2\alpha^*}{N}$ for all $S$.
%\end{theorem}

%\begin{proof}
%$\widehat{\mathcal{D}}(S) = \frac{1}{N} \sum\limits_{x} \mathcal{D}(x) (-1)^{Sx} = \frac{1}{N} ((\frac{1}{2}+ \alpha) - (\frac{1}{2} - \alpha)) = \frac{2\alpha}{N} \leq \frac{2\alpha^*}{N}$ 
%\end{proof}

\begin{definition}
\textit{Statistical Different} between two distributions $A$ and $B$ is defined as follow:
$$SD(A,B) = \frac{1}{2} \sum\limits_{i} |a_i - b_i |$$
\end{definition}
\begin{theorem}
Let $\mathcal{D}$ be a small bias distribution with $\widehat{\mathcal{D}}(S) \leq \frac{\alpha}{N}$ for all $S$, let $\mathcal{M}$ be a min entropy source such that $H_\infty(\mathcal{M}) \geq k$, and let $\mathcal{U}$ be  the uniform distribution over $n$-bits string. Then $$SD(\mathcal{D} \oplus \mathcal{M}, \mathcal{U}) \leq \frac{\alpha \sqrt{N}}{2^{1+k/2}}$$
\end{theorem}

\begin{proof} 
\begin{align*}
SD(\mathcal{D} \oplus \mathcal{M}, \mathcal{U})
&= \frac{1}{2} \sum\limits_i |(\mathcal{D} \oplus \mathcal{M})(i) - \mathcal{U}(i) | \\
& \leq \frac{1}{2}  \sqrt{N \sum\limits_i [(\mathcal{D} \oplus \mathcal{M})(i) - \mathcal{U}(i)]^2} \\
&= \frac{1}{2}  \sqrt{N^2 \cdot \| (\mathcal{D} \oplus \mathcal{M}) - \mathcal{U} \|_2^2} \\
& = \frac{N}{2} \sqrt{\sum\limits_{S} 
	[\widehat{\mathcal{D} \oplus \mathcal{M}}(S) - \mathcal{U}(S)]^2} \\
& = \frac{N}{2} \sqrt{\sum\limits_{S \neq \emptyset} 
	\widehat{\mathcal{D} \oplus \mathcal{M}}(S)^2} \\
\end{align*}
%&= \frac{N}{2} \sqrt{\sum\limits_{S \neq \emptyset}
%	N^2 \cdot \widehat{\mathcal{D}}(S)^2 \cdot \widehat{\mathcal{M}}(S)^2} \\
%&\leq \frac{N}{2} \sqrt{\sum\limits_{S \neq \emptyset}
%	N^2 \cdot (\frac{\alpha}{N})^2 \cdot \widehat{\mathcal{M}}(S)^2} \\
%&= \frac{\alpha N}{2} \sqrt{\sum\limits_{S \neq \emptyset} \widehat{\mathcal{M}}(S)^2} \\
%\leq 
By convolution,
\begin{align*}
\sum\limits_{S \neq \emptyset} \widehat{\mathcal{D} \oplus \mathcal{M}}(S)^2 
&= \sum\limits_{S \neq \emptyset} N^2 \cdot \widehat{\mathcal{D} * \mathcal{M}} (S)^2 \\
&= N^2 \sum\limits_{S \neq \emptyset} \widehat{\mathcal{D}}(S)^2 \cdot \widehat{\mathcal{M}}(S)^2 \\
& \leq N^2 \cdot  \sum\limits_{S \neq \emptyset} (\frac{\alpha}{N})^2 \cdot \widehat{\mathcal{M}}(S)^2 \\
&= \alpha^2 \cdot \sum\limits_{S \neq \emptyset} \widehat{\mathcal{M}}(S)^2 \\
& \leq \frac{\alpha^2}{N \cdot 2^k}
\end{align*}
Hence, 
$$SD(\mathcal{D} \oplus \mathcal{M}, \mathcal{U}) 
\leq \frac{\alpha \sqrt{N}}{2^{1+k/2}}$$
\end{proof}

\appendix

\chapter{}
\section{Dual of a Vector Space}
\begin{definition} [Dual space]
Let $V$ be a subspace of $\{0,1\}^n$. 
We define the dual of V as 
$V^\perp = \{ x \in \{ 0,1 \}^n | x \cdot v = 0 \ \forall v \in V \}$.
\end{definition}

\begin{theorem}
$V^\perp$ is a subspace of $\{ 0,1 \}^n$.
\end{theorem}

\begin{proof}
For any $x, y \in V^\perp, a \in \{0,1\}, (a \cdot x+y)\cdot v = a \cdot (x \cdot v) + y \cdot v = 0 + 0 = 0$. 
\end{proof} 

\begin{lemma} $\sum \limits_{i: \text{even}}^t {n \choose i} = 
\sum \limits_{i: \text{odd}}^t {n \choose i} = 2^{t-1} $.
\end{lemma}

\begin{theorem} For any subspace $V$ of dimension $k$ of $\{ 0,1 \}^n$, 
there exists a unique dual space $V^\perp$ of dimension $(n-k)$.
\end{theorem}

\begin{proof}
We will show that $|V^\perp| = 2^{n-k}$ by induction on $k$.\\
If $k=0$, then $V = \{ \textbf{0}\}$. Clearly, $V^\perp = \{ 0, 1 \}^n$. \\
If $k=1$, let $V = \{ \vec{0}, v \}$. 
Suppose the number of $v_i = 1$ is $t$, 
then the number of $x$ such that $x \cdot v = 0$ is 
$\sum \limits_{i: 2|t-i} {n \choose i} 2^{n-t} =
2^{t-1} \cdot 2^{n-t} = 2^{n-1}$ by Lemma A.3. \\
Suppose that there exists a unique orthogonal subspace $V^\perp$ 
of dimension $(n-k+1)$ for any subspace $V$ of dimension $k-1$ of $\{ 0,1 \}^n$, where $k \geq 2$. \\
Let $V=\langle v_1, v_2,..., v_{k} \rangle$, 
$S_1 = \langle v_1, v_2,..., v_{k-1}\rangle$, 
and $S_2 = \langle v_{k}\rangle$.  Then, $V^\perp = S_1^\perp \cap S_2^\perp$. \\
Suppose $dim(V^\perp) = t$. We want to show $t = n-k$.\\
By induction hypothesis, $dim(S_1^\perp) = n-k+1$ 
and $dim(S_2^\perp) = n-1$. \\
If $t \leq n-k-1$, then we need $[(n-k+1) - t]$ independent vectors to cover $S_1^\perp$ from extending $V^\perp$, and we need $[(n-1) -t]$ independent vectors to cover $S_2^\perp$ from extending $V^\perp$. Since $S_1^\perp \cup S_2^\perp \subseteq \{ 0,1\}^n$, we must have $[(n-k+1) - t] + [(n-1) -t] + t \leq n$, which is equivalent to $t \geq n-k$, contradiction.\\
If $t \geq n-k+1$, then $S_1^\perp \subseteq S_2^\perp$,
this is impossible since $v_k$ is independent from $v_1, v_2, ..., v_{k-1}$. \\
Thus, $t = n-k$. So $|V^\perp| = 2^{n-k}$. 
\end{proof}

\end{document}

